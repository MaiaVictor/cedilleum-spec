\documentclass{article}

\usepackage{amsmath,amssymb,amsthm}
%\usepackage{unicode-math}
\usepackage{url}
\usepackage{fullpage}

\usepackage{subcaption}
\usepackage{cedilleverbatim}
\usepackage{proof}
\DeclareUnicodeCharacter{03BC}{\ensuremath{\mu}}
\DeclareUnicodeCharacter{21A6}{\ensuremath{\mapsto}}
\DeclareUnicodeCharacter{25CF}{\ensuremath{\bullet}}
\DeclareUnicodeCharacter{1D48C}{\ensuremath{\kappa}}

% useful macros
\newcommand{\ann}[2]{#1\! : \! #2}
\newcommand{\abs}[4]{{#1}\, #2\! : \! #3.\, #4}
\newcommand{\absu}[3]{{#1}\, #2.\, #3}
\mathchardef\mhyph="2D % Define a "math hyphen"

% - type inference
\newcommand{\decdir}{\vdash_{\delta}}
\newcommand{\decsyn}{\vdash_{\Uparrow}}
\newcommand{\decchk}{\vdash_{\Downarrow}}

\begin{document}

\title{The Cedilleum Language Specification \\ \large Syntax, Typing, Reduction,
  and Elaboration }

\author{Christopher Jenkins}

\maketitle

\section{Syntax}

\paragraph{Identifiers}
\begin{figure}[h]
  \[
    \begin{array}{llll}
      id & &
      & \textnormal{identifiers for definitions}
      \\ u & &
      & \textnormal{term variables}
      \\ X & &
      & \textnormal{type variables}
      \\ 𝒌 & &
      & \textnormal{kind variables}
      \\ x & ::= & id\ |\ u\ |\ X\
      & \textnormal{non-kind variables}
      \\ y & ::= & x\ |\ 𝒌 & \text{all variables}
    \end{array}
  \]
  \caption{Identifiers}
  \label{fig:identifiers}
\end{figure}

Figure \ref{fig:identifiers} gives the metavariables used in our grammar for
identifiers. We consider all identifiers as coming from two distinct lexical
``pools'' -- regular identifiers (consisting of identifiers $id$ given for
modules and definitions, term variables $u$, and type variables $X$) and kind
identifiers $\kappa$. In Cedilleum source files (as in the parent language Cedille)
kind variables should be literally prefixed with $\kappa$ -- the suffix can be
any string that would by itself be a legal non-kind identifier.

\paragraph{Untyped Terms}
\begin{figure}[h]
  \[
    \begin{array}{llll}
      p
      & ::= & u
      & \text{variables}
      \\ & & \absu{\textbf{λ}}{u}{p}
      & \text{functions}
      \\ & & p\ p'
      & \text{applications}
      \\ & & \textbf{μ}\langle u_{rec}\ u_{out}\ u_{z} \rangle\ e_s\ \textbf{\{}
             pcase^* \textbf{\}}
      & \text{fixed-point and pattern matching}
      \\ & & \textbf{μ'}\ p_s\ p^*
      & \text{auxiliary pattern matching}
      \\ \\ pcase
      & ::= & \textbf{\textbar}\ u\ u^* \mapsto p
    \end{array}
  \]
  \caption{Untyped terms}
  \label{fig:pure-terms}
\end{figure}

The grammar of pure (untyped) terms the untyped λ-calculus augmented with a
primitives for combination fixed-point and pattern-matching definitions (and an
auxiliary pattern-matching construct).

\begin{figure}[h]
  \[
    \begin{array}{llll}
      % module stuff
      \\ mod
      & ::= & \textbf{module}\ id\ \textbf{.}\ imprt^*\ cmd^*\
      & \textnormal{module declarations}
      \\ imprt
      & ::= & \textbf{import}\ id\ \textbf{.}
      & \textnormal{module imports}
      \\ cmd
      & ::= & defTermOrType
      & \textnormal{definitions}
      \\ & & defDataType
      \\ & & defKind
      % definitions
      \\ 
      \\ defTermOrType
      & ::= & id\ checkType^?\ \textbf{=}\ t\ \textbf{.}
      & \textnormal{term definition}
      \\ & & id\ \textbf{:}\ K\ \textbf{=}\ T\ \textbf{.}
      & \textnormal{type definition}
      \\ defKind
      & ::= & 𝒌\ \textbf{=}\ K
      & \text{kind definition}
      \\ defDataType
      & ::= & \textbf{data}\ id\ param^*\ \textbf{:}\ K\ \textbf{=}\
              constr^*\ \textbf{.}
      & \textnormal{datatype definitions}
     % auxilliary categories for definitions
      \\ 
      \\ checkType
      & ::= & \textbf{:}\ T
      & \textnormal{annotation for term definition}
      \\ param
      & ::= & \textbf{(}x\ \textbf{:}\ C \textbf{)}
      \\ constr
      & ::= & \textbf{\textbar}\ id\ \textbf{:}\ T
    \end{array}
  \]
  \caption{Modules and definitions}
  \label{fig:mods-defs}
\end{figure}

\paragraph{Modules and Definitions}
All Cedilleum source files start with production $mod$, which consists of a module
declaration, a sequence of import statements which bring into scope definitions
from other source files, and a sequence of \textit{commands} defining terms,
types, and kinds. As an illustration, consider the first few lines of a
hypothetical \texttt{list.ced}:

\begin{verbatim}
module list .

import nat .
\end{verbatim}

\noindent Imports are handled first by consulting a global options files
known to the Cedilleum compiler (on *nix systems \verb|~/.cedille/options|)
containing a search path of directories, and next (if that fails) by searching
the directory containing the file being checked.

Term and type definitions are given with an identifier, a classifier (type or
kind, resp.) to check the definition against, and the definition. For term
definitions, giving classifier (i.e. the type) is optional. As an example,
consider the definitions for the type of Church-encoded lists and two variants
of the nil constructor, the first with a top-level type annotation and the
second with annotations sprinkled on binders:

\begin{verbatim}
cList : ★ ➔ ★
      = λ A : ★ . ∀ X : ★ . (A ➔ X ➔ X) ➔ X ➔ X .

cNil  : ∀ A : ★ . cList · A
      = Λ A . Λ X . λ c . λ n . n .
cNil' = Λ A : ★ . Λ X : ★ . λ c : A ➔ X ➔ X . λ n : X . n .
\end{verbatim}

Kind definitions are given without classifiers (all kinds have super-kind
$\Box$), e.g. \verb;𝒌func = ★ ➔ ★;

Inductive datatype definitions take a set of \textit{parameters} (term and type
variables which remain constant throughout the definition) well as a set of
\textit{indices} (term and type variables which \textit{can} vary), followed by
zero or more constructors. Each constructor begins with ``\textbf{\textbar}''
(though the grammar can be relaxed so that the first of these is optional) and
then an identifier and type is given. As an example, consider the following two
definitions for lists and vectors (length-indexed lists).

\begin{verbatim}
data List (A : ★) : ★ =
  | nil  : List
  | cons : A ➔ List ➔ List
  .
data Vec (A : ★) : Nat ➔ ★ =
  | vnil  : Vec Z
  | vcons : ∀ n : Nat . A ➔ Vec n ➔ Vec (S n)
  .
\end{verbatim}

\paragraph{Types and Kinds}
\begin{figure}[h]
  \[
    \begin{array}{llll}
      S
      & ::= & \square & \text{sole super-kind}
      \\ & & K & \text{kinds}
      \\ C
      & ::= & K & \text{types}
      \\ & & T & \text{types}
      \\ K
      & ::= & \textbf{Π}\ x\ \textbf{:}\ C\ \textbf{.}\ K
      & \textnormal{explicit product}
      \\ & & C\ \textbf{➔}\ K
      & \textnormal{kind arrow}
      \\ & & \textbf{★}
      & \text{the kind of types that classify terms}
      \\ & & \textbf{(}K\textbf{)}
      \\ 
      \\ T
      & ::= & \textbf{Π}\ x\ \textbf{:}\ T\ \textbf{.}\ T
         & \textnormal{explicit product}
      \\ & &  \textbf{∀}\ x\ \textbf{:}\ C\ \textbf{.}\ T
         & \textnormal{implicit product}
      \\ & &  \textbf{λ}\ x\ \textbf{:}\ C\ \textbf{.}\ T
         & \textnormal{type-level function}
      \\ & & T\ \textbf{➾}\ T'
         & \textnormal{arrow with erased domain}
      \\ & & T\ \textbf{➔}\ T'
         & \textnormal{normal arrow type}
      \\ & & T\ \textbf{·}\ T'
         & \text{application to another type}
      \\ & & T\ t
         & \text{application to a term}
      \\ & & \textbf{\{}\ p\ ≃\ p' \textbf{\}}
         & \textnormal{untyped equality}
      \\ & & \textbf{(}T\textbf{)}
      \\ & & X
         & \text{type variable}
      \\ & & \bullet
         & \text{hole for incomplete types}
    \end{array}
  \]
  \caption{Kinds and types}
  \label{fig:kinds-types}
\end{figure}

In Cedilleum, the expression language is stratified into three main ``classes'':
kinds, types, and terms. Kinds and types are listed in Figure
\ref{fig:kinds-types} and terms are listed in Figure \ref{fig:ann-terms} along
with some auxiliary grammatical categories -- $s$ groups terms and types
together as ``subjects'' resp. of typing and kinding, and $C$ groups types and
kinds together as classifiers. In both of these figures, the constructs forming
expressions are listed from lowest to highest precedence -- ``abstractors''
($\lambda\ \Lambda\ \Pi\ \forall$) bind most loosely and parentheses most
tightly. Associativity is as-expected, with arrows (➔ ➾) and applications being
left-associative and abstractors being right-associative.

% TODO cite
The language of kinds and types is similar to that found in the Calculus of
Implicit Constructions\footnote{Cite}. Kinds are formed by dependent and
non-dependent products (Π and ➔) and a base kind for types which can classify
terms (★). Types are also formed by the usual (dependent and non-dependent)
products (Π and ➔) and also \textit{implicit} products (∀ and ➾) which quantify
over erased arguments (that is, arguments that disappear at run-time).
Π-products are only allowed to quantify over terms as all types occurring in
terms are erased at run-time, but ∀-products can quantify over types
\textit{and} terms because terms can be erased. Meanwhile, non-dependent
products (➔ and ➾) can only ``quantify'' over terms because non-dependent type
quantification does not seem particularly useful. Besides these, Cedilleum
features type-level functions and applications (with term and type arguments),
and a primitive equality type for untyped terms. Last of all is the ``hole''
type (●) for writing partial type signatures or incomplete type applications.
There are term-level holes as well, and together the two are intended to help
facilitate ``hole-driven development'': any hole automatically generates a type
error and provides the user with useful contextual information.

We illustrate with another example: what follows is a module stub for
\textbf{DepCast} defining dependent casts -- intuitively, functions from $a : A$
to $B\ a$ that are also equal\footnote{Module erasure, discussed below} to
identity -- where the definitions \texttt{CastE} and \texttt{castE} are
incomplete.

\begin{verbatim}
module DepCast .

CastE ◂ Π A : ★ . (A ➔ ★) ➔ ★ = ● .

castE ◂ ∀ A : ★ . ∀ B : A ➔ ★ . CastE · A · B ➾ Π a : A . B a = ● .
\end{verbatim}

\paragraph{Terms}
\begin{figure}[h]
  \[
    \begin{array}{llll}
      s
      & ::= & t & \text{term}
      \\ & & T & \text{type}
      \\ t
      & ::= & \textbf{λ}\, x\ class^?\! \textbf{.}\, t
      & \textnormal{normal abstraction}
      \\ & & \textbf{Λ}\, x\ class^?\! \textbf{.}\, t
      & \textnormal{erased abstraction}
      \\ & & \textbf{[}\ defTermOrType\ \textbf{]}\ \textbf{-}\ t
      & \text{let definitions}
      \\ & & \textbf{ρ}\ t\ \textbf{-}\ t'
      & \text{equality elimination by rewriting}
      \\ & & \textbf{φ}\ t\ \textbf{-}\ t'\ \textbf{\{} t'' \textbf{\}}
      & \text{type cast}
      \\ & & \textbf{χ}\ T^?\ \textbf{-}\ t
      & \text{check a term against a type}
      \\ & & \textbf{δ}\ \textbf{-}\ t
      & \text{ex falso quodlibet}
      \\ & & \textbf{θ}\ t\ t'^*
      & \text{elimination with a motive}
      \\ & & t\ t'
      & \text{applications}
      \\ & & t\ \textbf{-}t'
      & \text{application to an erased term}
      \\ & & t\ \textbf{·}T
      & \text{application to a type}
      \\ & & \textbf{β}\ \textbf{\{} t \textbf{\}}
      & \textnormal{reflexivity of equality}
      \\ & & \textbf{ς}\ t
      & \textnormal{symmetry of equality}
      \\ & & \textbf{μ}\langle u,X,u',u'' \rangle\ t\ motive^?\ \textbf{\{}\ case^*\ \textbf{\}}
      & \textnormal{type-guarded pattern match and fixpoint}
      \\ & & \textbf{μ'}\ t\ motive^?\ \textbf{\{}\ case^*\ \textbf{\}}
      & \text{auxiliary pattern match}
      \\ & & u
      & \text{term variable}
      \\ & & \textbf{(}t\textbf{)}
      \\ & & \bullet
      & \text{hole for incomplete term}
      \\ case
      & ::= & \textbf{\textbar}\ id\ vararg^*\ \textbf{↦}\ t
      & \text{pattern-matching cases}
      \\ vararg
      & ::= & u
      & \text{normal constructor argument}
      \\ & & \textbf{-}u
      & \text{erased constructor argument}
      \\ & & \textbf{·}X
      & \text{type constructor argument}
      \\ class
      & ::= & \textbf{:}\ C
      \\ motive
      & ::= & \textbf{@}\ T
      & \textnormal{motive for induction}
    \end{array}
  \]
  \caption{Annotated Terms}
  \label{fig:ann-terms}
\end{figure}

Terms can be explicit and implicit functions (resp. indicated by λ and Λ) with
optional classifiers for bound variables, let-bindings, applications $t\ t'$,
$t\ \mhyph t'$, and $t\ \cdot T$ (resp. to another term, an erased term, or a
type). In addition to this there are a number of useful operators that will be
discussed in more detail (and whose purpose will become more apparent) in Section
\ref{sec:type-system}. 

\section{Erasure}

\begin{figure}[h]
  \[
  \begin{array}{lll}
       |x| & = & x 
    \\ |\star| & = & \star 
    \\ |\Box| & = & \Box 
    \\ |\beta\ \{t\}| & = & |t|
    \\ |\delta\ t| & = & |t|
    \\ |\chi\ T^? \textbf{-}\ t| & = & |t| 
    \\ |\theta\ t\ t'^*| & = & |t|\ |t'^*| 
    \\ |\varsigma\ t| & = & |t|
    \\ |t\ t'| & = & |t|\ |t'|
    \\ |t\ \mhyph t'| & = & |t| 
    \\ |t\ \cdot T| & = & |t| 
    \\ |\rho\ t\ \mhyph\ t'| & = & |t'| 
    \\ |\abs{\forall}{x}{C}{C'}| & = & \abs{\forall}{x}{|C|}{|C'|}
    \\ |\abs{\Pi}{x}{C}{C'}| & = & \abs{\Pi}{x}{|C|}{|C'|}
    \\ |\abs{\lambda}{u}{T}{t}| & = &  \absu{\lambda}{u}{|t|} 
    \\ |\absu{\lambda}{u}{t}| & = &  \absu{\lambda}{u}{|t|} 
    \\ |\abs{\lambda}{X}{K}{C}| & = &  \abs{\lambda}{X}{|K|}{|C|} 
    \\ |\abs{\Lambda}{x}{C}{t}| & = &  |t| 
    \\ |\phi\ t\ \mhyph\ t'\ \{t''\}| & = & |t''| 
    \\ |[ x = t : T]|\ \mhyph\ t' | & = & (\absu{\lambda}{x}{|t'|})\ |t|
    \\ |[X = T : K]\ \mhyph\ t | & = & |t| 
    \\ |\{ t \simeq t' \}|| & = & \{ |t| \simeq |t'| \}
    \\ |\mu\langle u,X,u',u'' \rangle\ t\ motive^?\ \textbf{\{} case^* \textbf{\}}|
           & = & \mu\langle u,u',u'' \rangle\ t\ \textbf{\{}|case^*|\textbf{\}}
    \\ |\mu'\ t\ motive^?\ \textbf{\{} case^* \textbf{\}}| & = & \mu'\ t\
       \textbf{\{}|case^*|\textbf{\}}
    \\ \\ |id\ vararg^* \mapsto t| & = & id\ |vararg^*|\ \mapsto |t|
    \\ 
    \\ |\mhyph u| & = & 
    \\ |\cdot T|  & = &
  \end{array}
  \]
  \caption{Erasure for annotated terms}
  \label{fig:eraser}
\end{figure}

The definition of the erasure function given in Figure \ref{fig:eraser} takes
the annotated terms from Figures \ref{fig:kinds-types} \ref{fig:ann-terms} to
the untyped terms of Figure \ref{fig:pure-terms}. The last two equations
indicate that any type or erased arguments in the the zero or more $vararg$'s of
pattern-match case are indeed erased.

\section{Type System}
\label{sec:type-system}

\begin{figure}[h]
  \[
    \begin{array}{lll}
      \infer{\Gamma\vdash \star : \Box}{\ }
      & \infer
        { \Gamma\vdash\abs{\Pi}{y}{C}{C'} : S'}
        { \Gamma \vdash C : S
        \quad
        \Gamma,y:C\vdash C' : S'
        \quad \textit{Var}(y,S)
        }
      & \infer
        {\Gamma\vdash\abs{\forall}{y}{C}{C'} : \star}
        {\Gamma \vdash C : S
        \quad \Gamma,y:C\vdash C' : \star
        \quad \textit{Var}(y,S)
        }
      \\
      \\ \infer
      { \Gamma \vdash \{p \simeq p' \} : \star}
      { FV(p\ p') \subseteq dom(\Gamma) }
      & \infer
        { \Gamma \vdash \kappa : \Gamma(\kappa)}
        { }
      & \infer
        { \Gamma \vdash X : \Gamma(X)}
        { }
      \\
      \\ \infer
      { \Gamma \vdash \abs{\lambda}{x}{C}{T} : \abs{\Pi}{x}{C}{K}}
      { \Gamma \vdash \abs{\Pi}{x}{C}{K} : \square
      \quad \Gamma, \ann{x}{C} \vdash T : K
      }
      & \infer
        { \Gamma \vdash T\ \cdot T' : [T'/x] K'}
        { \Gamma \vdash T : \abs{\Pi}{x}{K}{K'}
        \quad \Gamma \vdash T' : K}
      & \infer
       { \Gamma \vdash T\ t : [t/x] K}
        { \Gamma \vdash T : \abs{\Pi}{x}{T'}{K}
        \quad \Gamma \decchk t : T' }
    \end{array}
  \]
  \caption{Kind and sort checking for Cedilleum}
\end{figure}

\begin{figure}[h]
  \[
    \begin{array}{lll}
      \infer
      { \Gamma \decdir u : \Gamma(u)}{}
      & \infer
        { \Gamma \decdir \abs{\lambda}{x}{T}{t} : \abs{\Pi}{x}{T}{T'}}
        { \Gamma \vdash T : K
        \quad \Gamma, \ann{x}{T} \decdir t : T'}
      & \infer
        { \Gamma \decdir \absu{\lambda}{x}{t} : \abs{\Pi}{x}{T}{T'}}
        { \Gamma, \ann{x}{T} \decdir t : T'}
      \\
      \\ \infer
      { \Gamma \decdir \abs{\Lambda}{x}{C}{t} : \abs{\forall}{x}{C}{T}}
      { \Gamma \vdash C : S
      \quad x \notin FV(|t|)
      \quad \Gamma, \ann{x}{C} \decdir t : T
      }
      & \infer
        { \Gamma \decchk \absu{\Lambda}{x}{t} : \abs{\forall}{x}{C}{T}}
        { x \notin FV(|t|)
        \quad \Gamma, \ann{x}{C} \decdir t : T
        }
      & \infer
        { \Gamma \decdir t\ t' : [t'/x]T}
        { \Gamma \decsyn t : \abs{\Pi}{x}{T'}{T}
        \quad \Gamma \decchk t' : T'}
      \\
      \\ \infer
      { \Gamma \decdir t\ \cdot T : [T/X]T'}
      { \Gamma \decsyn t : \abs{\forall}{X}{K}{T'}
      \quad \Gamma \vdash T : K}
      & \infer
        { \Gamma \decdir t\ \mhyph t' : [t'/x]T}
        { \Gamma \decsyn t : \abs{\forall}{x}{T'}{T}
        \quad \Gamma \decchk t' : T'}
    \end{array}
  \]
  \caption{Type checking for Cedilleum}
\end{figure}

% TODO kind-variables... two different rules or Var check?

\end{document}