\documentclass{article}


% \usepackage{todonotes}
\usepackage{amsmath,amssymb,amsthm}
%\usepackage{unicode-math}
\usepackage{url}
\usepackage{fullpage}

\usepackage{subcaption}
\usepackage{cedilleverbatim}
\usepackage{proof}

\usepackage{stmaryrd}
% \usepackage{unicode-math}

\DeclareUnicodeCharacter{03BC}{\ensuremath{\mu}}
\DeclareUnicodeCharacter{21A6}{\ensuremath{\mapsto}}
\DeclareUnicodeCharacter{25CF}{\ensuremath{\bullet}}
\DeclareUnicodeCharacter{1D48C}{\ensuremath{\kappa}}

% useful macros
\newcommand{\ann}[2]{#1\! : \! #2}
\newcommand{\abs}[4]{{#1}\, #2\! : \! #3.\, #4}
\newcommand{\absu}[3]{{#1}\, #2.\, #3}
\mathchardef\mhyph="2D % Define a "math hyphen"
\newcommand{\indast}[4]{\texttt{Ind}_{#1} [#2] (#3 := #4)}
\newcommand{\lowerc}[1]{\lfloor {#1} \rfloor}
\newcommand{\lenc}[1]{\|#1\|}
\newcommand{\vars}[1]{{\overline{#1}}}

% - type inference
\newcommand{\decdir}{\vdash_{\delta}}
\newcommand{\decsyn}{\vdash_{\Uparrow}}
\newcommand{\decchk}{\vdash_{\Downarrow}}

% - inductive
\newcommand{\mufix}[3]{μ\ #1\ .\ #2\ \{ #3 \}}
\newcommand{\mumat}[2]{μ'\ #1\ \{#2\} }
\newcommand{\wfpat}[4]{WF\!\mhyph\!Pat(#1,#2,#3,#4)}
\newcommand{\llbrace}{\{\!\{}
\newcommand{\rrbrace}{\}\!\}}
\newcommand{\piforall}{^{\Pi}_{\forall}}

\begin{document}

\title{The Cedilleum Language Specification \\ \large Syntax, Typing, Reduction,
  and Elaboration }

\author{Christopher Jenkins}

\maketitle

\section{Syntax}
\label{sec:syntax}

\paragraph{Identifiers}
\begin{figure}[h]
  \[
    \begin{array}{llll}
      id & &
      & \textnormal{identifiers for definitions}
      \\ u,c & &
      & \textnormal{term variables}
      \\ X & &
      & \textnormal{type variables}
      \\ 𝒌 & &
      & \textnormal{kind variables}
      \\ x & ::= & id\ |\ u\ |\ X\
      & \textnormal{non-kind variables}
      \\ y & ::= & x\ |\ 𝒌 & \text{all variables}
    \end{array}
  \]
  \caption{Identifiers}
  \label{fig:identifiers}
\end{figure}

Figure \ref{fig:identifiers} gives the metavariables used in our grammar for
identifiers. We consider all identifiers as coming from two distinct lexical
``pools'' -- regular identifiers (consisting of identifiers $id$ given for
modules and definitions, term variables $u$, and type variables $X$) and kind
identifiers $\kappa$. In Cedilleum source files (as in the parent language Cedille)
kind variables should be literally prefixed with $\kappa$ -- the suffix can be
any string that would by itself be a legal non-kind identifier. For example,
\texttt{myDef} is a legal term / type variable and a legal name for a
definition, whereas \texttt{𝒌myDeff} is only legal as a kind definition.

\paragraph{Untyped Terms}
\begin{figure}[h]
  \[
    \begin{array}{llll}
      p
      & ::= & u
      & \text{variables}
      \\ & & \absu{\textbf{λ}}{u}{p}
      & \text{functions}
      \\ & & p\ p
      & \text{applications}
      \\ & & \mufix{u}{p}{pcase^*}
      & \text{fixed-point and pattern matching}
      \\ & & \textbf{μ'}\ p\ \textbf{\{} pcase^* \textbf{\}}
      & \text{simple pattern matching}
      \\ \\ pcase
      & ::= & \textbf{\textbar}\ u\ u^* \mapsto p
    \end{array}
  \]
  \caption{Untyped terms}
  \label{fig:pure-terms}
\end{figure}

The grammar of pure (untyped) terms the untyped λ-calculus augmented with a
primitives for combination fixed-point and pattern-matching definitions (and an
auxiliary pattern-matching construct).

\begin{figure}[h]
  \[
    \begin{array}{llll}
      % module stuff
      \\ mod
      & ::= & \textbf{module}\ id\ \textbf{.}\ imprt^*\ cmd^*\
      & \textnormal{module declarations}
      \\ imprt
      & ::= & \textbf{import}\ id\ \textbf{.}
      & \textnormal{module imports}
      \\ cmd
      & ::= & defTermOrType
      & \textnormal{definitions}
      \\ & & defDataType
      \\ & & defKind
      % definitions
      \\ 
      \\ defTermOrType
      & ::= & id\ checkType^?\ \textbf{=}\ t\ \textbf{.}
      & \textnormal{term definition}
      \\ & & id\ \textbf{:}\ K\ \textbf{=}\ T\ \textbf{.}
      & \textnormal{type definition}
      \\ defKind
      & ::= & 𝒌\ \textbf{=}\ K
      & \text{kind definition}
      \\ defDataType
      & ::= & \textbf{data}\ id\ param^*\ \textbf{:}\ K\ \textbf{=}\
              constr^*\ \textbf{.}
      & \textnormal{datatype definitions}
     % auxilliary categories for definitions
      \\ 
      \\ checkType
      & ::= & \textbf{:}\ T
      & \textnormal{annotation for term definition}
      \\ param
      & ::= & \textbf{(}x\ \textbf{:}\ C \textbf{)}
      \\ constr
      & ::= & \textbf{\textbar}\ id\ \textbf{:}\ T
    \end{array}
  \]
  \caption{Modules and definitions}
  \label{fig:mods-defs}
\end{figure}

\paragraph{Modules and Definitions}
All Cedilleum source files start with production $mod$, which consists of a module
declaration, a sequence of import statements which bring into scope definitions
from other source files, and a sequence of \textit{commands} defining terms,
types, and kinds. As an illustration, consider the first few lines of a
hypothetical \texttt{list.ced}:

\begin{verbatim}
module list .

import nat .
\end{verbatim}

\noindent Imports are handled first by consulting a global options files
known to the Cedilleum compiler (on *nix systems \verb|~/.cedille/options|)
containing a search path of directories, and next (if that fails) by searching
the directory containing the file being checked.

Term and type definitions are given with an identifier, a classifier (type or
kind, resp.) to check the definition against, and the definition. For term
definitions, giving classifier (i.e. the type) is optional. As an example,
consider the definitions for the type of Church-encoded lists and two variants
of the nil constructor, the first with a top-level type annotation and the
second with annotations sprinkled on binders:

\begin{verbatim}
cList : ★ ➔ ★
      = λ A : ★ . ∀ X : ★ . (A ➔ X ➔ X) ➔ X ➔ X .

cNil  : ∀ A : ★ . cList · A
      = Λ A . Λ X . λ c . λ n . n .
cNil' = Λ A : ★ . Λ X : ★ . λ c : A ➔ X ➔ X . λ n : X . n .
\end{verbatim}

Kind definitions are given without classifiers (all kinds have super-kind
$\Box$), e.g. \verb;𝒌func = ★ ➔ ★;

Inductive datatype definitions take a set of \textit{parameters} (term and type
variables which remain constant throughout the definition) well as a set of
\textit{indices} (term and type variables which \textit{can} vary), followed by
zero or more constructors. Each constructor begins with ``\textbf{\textbar}''
(though the grammar can be relaxed so that the first of these is optional) and
then an identifier and type is given. As an example, consider the following two
definitions for lists and vectors (length-indexed lists).

\begin{verbatim}
data Bool : ★ =
  | tt : Bool
  | ff : Bool
  .

data Nat : ★ =
  | zero : Nat
  | suc  : Nat ➔ Nat
  .

data List (A : ★) : ★ =
  | nil  : List
  | cons : A ➔ List ➔ List
  .
data Vec (A : ★) : Nat ➔ ★ =
  | vnil  : ∀ n : Nat . {n ≃ zero} ➾ Vec n
  | vcons : ∀ n : Nat . ∀ m : Nat . A ➔ Vec n ➔ {m ≃ suc n} ➾ Vec m
  .
\end{verbatim}

\paragraph{Types and Kinds}
\begin{figure}[h]
  \[
    \begin{array}{rlll}
      \text{Sorts } S
      & ::= & \square & \text{sole super-kind}
      \\ & & K & \text{kinds}
      \\ \text{Classifiers } C
      & ::= & K & \text{kinds}
      \\ & & T & \text{types}
      \\ \text{Kinds } K
      & ::= & \textbf{Π}\ x\ \textbf{:}\ C\ \textbf{.}\ K
      & \textnormal{explicit product}
      \\ & & C\ \textbf{➔}\ K
      & \textnormal{kind arrow}
      \\ & & \textbf{★}
      & \text{the kind of types that classify terms}
      \\ & & \textbf{(}K\textbf{)}
      \\ 
      \\ \text{Types } T
      & ::= & \textbf{Π}\ x\ \textbf{:}\ T\ \textbf{.}\ T
         & \textnormal{explicit product}
      \\ & &  \textbf{∀}\ x\ \textbf{:}\ C\ \textbf{.}\ T
         & \textnormal{implicit product}
      \\ & &  \textbf{λ}\ x\ \textbf{:}\ C\ \textbf{.}\ T
         & \textnormal{type-level function}
      \\ & & T\ \textbf{➾}\ T'
         & \textnormal{arrow with erased domain}
      \\ & & T\ \textbf{➔}\ T'
         & \textnormal{normal arrow type}
      \\ & & T\ \textbf{·}\ T'
         & \text{application to another type}
      \\ & & T\ t
         & \text{application to a term}
      \\ & & \textbf{\{}\ p\ ≃\ p' \textbf{\}}
         & \textnormal{untyped equality}
      \\ & & \textbf{(}T\textbf{)}
      \\ & & X
         & \text{type variable}
      \\ & & \bullet
         & \text{hole for incomplete types}
    \end{array}
  \]
  \caption{Kinds and types}
  \label{fig:kinds-types}
\end{figure}

In Cedilleum, the expression language is stratified into three main ``classes'':
kinds, types, and terms. Kinds and types are listed in Figure
\ref{fig:kinds-types} and terms are listed in Figure \ref{fig:ann-terms} along
with some auxiliary grammatical categories. In both of these figures, the
constructs forming expressions are listed from lowest to highest precedence --
``abstractors'' ($\lambda\ \Lambda\ \Pi\ \forall$) bind most loosely and
parentheses most tightly. Associativity is as-expected, with arrows (➔ ➾) and
applications being left-associative and abstractors being right-associative.

% TODO cite
The language of kinds and types is similar to that found in the Calculus of
Implicit Constructions\footnote{Cite}. Kinds are formed by dependent and
non-dependent products (Π and ➔) and a base kind for types which can classify
terms (★). Types are also formed by the usual (dependent and non-dependent)
products (Π and ➔) and also \textit{implicit} products (∀ and ➾) which quantify
over erased arguments (that is, arguments that disappear at run-time).
Π-products are only allowed to quantify over terms as all types occurring in
terms are erased at run-time, but ∀-products can quantify over types
\textit{and} terms because terms can be erased. Meanwhile, non-dependent
products (➔ and ➾) can only ``quantify'' over terms because non-dependent type
quantification does not seem particularly useful. Besides these, Cedilleum
features type-level functions and applications (with term and type arguments),
and a primitive equality type for untyped terms. Last of all is the ``hole''
type (●) for writing partial type signatures or incomplete type applications.
There are term-level holes as well, and together the two are intended to help
facilitate ``hole-driven development'': any hole automatically generates a type
error and provides the user with useful contextual information.

We illustrate with another example: what follows is a module stub for
\textbf{DepCast} defining dependent casts -- intuitively, functions from $a : A$
to $B\ a$ that are also equal\footnote{Module erasure, discussed below} to
identity -- where the definitions \texttt{CastE} and \texttt{castE} are
incomplete.

\begin{verbatim}
module DepCast .

CastE ◂ Π A : ★ . (A ➔ ★) ➔ ★ = ● .

castE ◂ ∀ A : ★ . ∀ B : A ➔ ★ . CastE · A · B ➾ Π a : A . B a = ● .
\end{verbatim}

\paragraph{Annotated Terms}
\begin{figure}[h]
  \[
    \begin{array}{rlll}
      \text{Subjects } s
      & ::= & t & \text{term}
      \\ & & T & \text{type}
      \\ \text{Terms } t
      & ::= & \textbf{λ}\, x\ class^?\! \textbf{.}\, t
      & \textnormal{normal abstraction}
      \\ & & \textbf{Λ}\, x\ class^?\! \textbf{.}\, t
      & \textnormal{erased abstraction}
      \\ & & \textbf{[}\ defTermOrType\ \textbf{]}\ \textbf{-}\ t
      & \text{let definitions}
      \\ & & \textbf{ρ}\ t\ \textbf{-}\ t'
      & \text{equality elimination by rewriting}
      \\ & & \textbf{φ}\ t\ \textbf{-}\ t'\ \textbf{\{} t'' \textbf{\}}
      & \text{type cast}
      \\ & & \textbf{χ}\ T\ \textbf{-}\ t
      & \text{check a term against a type}
      \\ & & \textbf{δ}\ \textbf{-}\ t
      & \text{ex falso quodlibet}
      \\ & & \textbf{θ}\ t\ t'^*
      & \text{elimination with a motive}
      \\ & & t\ t'
      & \text{applications}
      \\ & & t\ \textbf{-}t'
      & \text{application to an erased term}
      \\ & & t\ \textbf{·}T
      & \text{application to a type}
      \\ & & \textbf{β}\ \textbf{\{} t \textbf{\}}
      & \textnormal{reflexivity of equality}
      \\ & & \textbf{ς}\ t
      & \textnormal{symmetry of equality}
      \\ & & \mufix{u}{t\ motive^?}{case^*}
      & \textnormal{type-guarded pattern match and fixpoint}
      \\ & & \mumat{t\ motive^?}{case^*}
      & \text{auxiliary pattern match}
      \\ & & u
      & \text{term variable}
      \\ & & \textbf{(}t\textbf{)}
      \\ & & \bullet
      & \text{hole for incomplete term}
      \\ \\ case
      & ::= & \textbf{\textbar}\ c\ vararg^*\ \textbf{↦}\ t
      & \text{pattern-matching cases}
      \\ vararg
      & ::= & u
      & \text{normal constructor argument}
      \\ & & \textbf{-}u
      & \text{erased constructor argument}
      \\ & & \textbf{·}X
      & \text{type constructor argument}
      \\ class
      & ::= & \textbf{:}\ C
      \\ motive
      & ::= & \textbf{@}\ T
      & \textnormal{motive for induction}
    \end{array}
  \]
  \caption{Annotated Terms}
  \label{fig:ann-terms}
\end{figure}

Terms can be explicit and implicit functions (resp. indicated by λ and Λ) with
optional classifiers for bound variables, let-bindings, applications $t\ t'$,
$t\ \mhyph t'$, and $t\ \cdot T$ (resp. to another term, an erased term, or a
type). In addition to this there are a number of useful operators for
equaltional reasoning, type casting, providing annotations, and pattern
matching. Each operator will be discussed in more detail in Section
\ref{sec:type-system}, but a few concrete programs in Cedilleum are given below
merely to give a better idea of the syntax of the language.

\begin{verbatim}
isvnil : ∀ A : ★ . ∀ n : Nat . Vec · A n ➔ Bool
       = Λ A . Λ n . λ xs .
           μ' xs @(Λ n . λ xs . Bool)
              { | vnil -n -eq ↦ tt
                | vcons -n -m x xs -eq ↦ ff
              }
vlength : ∀ A : ★ . ∀ n : Nat . Vec · A n ➔ Nat
        = Λ A . Λ n . λ xs .
            μ len . xs @(Λ n . λ x . Nat)
              { | vnil -n -eq ↦ zero
                | vcons -n -m x xs -eq ↦ suc (len -n xs)
              }
\end{verbatim}

\section{Erasure}

\begin{figure}[h]
  \[
  \begin{array}{lll}
       |x| & = & x 
    \\ |\star| & = & \star 
    \\ |\Box| & = & \Box 
    \\ |\beta\ \{t\}| & = & |t|
    \\ |\delta\ t| & = & |t|
    \\ |\chi\ T^? \textbf{-}\ t| & = & |t| 
    \\ |\theta\ t\ t'^*| & = & |t|\ |t'^*| 
    \\ |\varsigma\ t| & = & |t|
    \\ |t\ t'| & = & |t|\ |t'|
    \\ |t\ \mhyph t'| & = & |t| 
    \\ |t\ \cdot T| & = & |t| 
    \\ |\rho\ t\ \mhyph\ t'| & = & |t'| 
    \\ |\abs{\forall}{x}{C}{C'}| & = & \abs{\forall}{x}{|C|}{|C'|}
    \\ |\abs{\Pi}{x}{C}{C'}| & = & \abs{\Pi}{x}{|C|}{|C'|}
    \\ |\abs{\lambda}{u}{T}{t}| & = &  \absu{\lambda}{u}{|t|} 
    \\ |\absu{\lambda}{u}{t}| & = &  \absu{\lambda}{u}{|t|} 
    \\ |\abs{\lambda}{X}{K}{C}| & = &  \abs{\lambda}{X}{|K|}{|C|} 
    \\ |\abs{\Lambda}{x}{C}{t}| & = &  |t| 
    \\ |\phi\ t\ \mhyph\ t'\ \{t''\}| & = & |t''| 
    \\ |[ x = t : T]|\ \mhyph\ t' | & = & (\absu{\lambda}{x}{|t'|})\ |t|
    \\ |[X = T : K]\ \mhyph\ t | & = & |t| 
    \\ |\{ t \simeq t' \}|| & = & \{ |t| \simeq |t'| \}
    \\ |\mufix{u,}{t\ motive^?}{case^*}|
           & = & \mufix{u}{|t|}{|case^*|}
    \\ |\mumat{t\ motive^?}{case^*}|
           & = & \mumat{|t|}{|case^*|}
    \\ \\ |id\ vararg^* \mapsto t| & = & id\ |vararg^*|\ \mapsto |t|
    \\ 
    \\ |\mhyph u| & = & 
    \\ |\cdot T|  & = &
  \end{array}
  \]
  \caption{Erasure for annotated terms}
  \label{fig:eraser}
\end{figure}

The definition of the erasure function given in Figure \ref{fig:eraser} takes
the annotated terms from Figures \ref{fig:kinds-types} and \ref{fig:ann-terms} to
the untyped terms of Figure \ref{fig:pure-terms}. The last two equations
indicate that any type or erased arguments in the the zero or more $vararg$'s of
pattern-match case are indeed erased. The additional constructs introduced in
the annotated term language such as β, φ, and ρ, are all erased to the language
of pure terms.

\section{Type System (sans Inductive Datatypes)}
\label{sec:type-system}

\begin{figure}[h]
  \caption{Contexts}
  \[
    \begin{array}{llll}
      \text{ Typing contexts } \Gamma
      & ::= & \emptyset\ |\ \ann{x}{C},\Gamma\ |\ \ann{x=s}{C},\Gamma
    \end{array}
  \]
\end{figure}
\begin{figure}[h!]
  \[ \small
    \begin{array}{lcr}
      \infer{\Gamma\vdash \star : \Box}{\ }
      & \infer
        { \Gamma\vdash\abs{\Pi}{y}{C}{C'} : S'}
        { \Gamma \vdash C : S
        \quad
        \Gamma,y:C\vdash C' : S'
%        \quad \textit{Var}(y,S)
        }
      & \infer
        {\Gamma\vdash\abs{\forall}{y}{C}{C'} : \star}
        {\Gamma \vdash C : S
        \quad \Gamma,y:C\vdash C' : \star
%        \quad \textit{Var}(y,S)
        }
      \\
      \\ \infer
      { \Gamma \vdash \{p \simeq p' \} : \star}
      { FV(p\ p') \subseteq dom(\Gamma) }
      & \infer
        { \Gamma \vdash \kappa : \Gamma(\kappa)}
        { }
      & \infer
        { \Gamma \vdash X : \Gamma(X)}
        { }
      \\
      \\ \infer
      { \Gamma \vdash \abs{\lambda}{x}{C}{T} : \abs{\Pi}{x}{C}{K}}
      { \Gamma \vdash \abs{\Pi}{x}{C}{K} : \square
      \quad \Gamma, \ann{x}{C} \vdash T : K
      }
      & \infer
        { \Gamma \vdash T\ \cdot T' : [T'/x] K'}
        { \Gamma \vdash T : \abs{\Pi}{x}{K}{K'}
        \quad \Gamma \vdash T' : K}
      & \infer
       { \Gamma \vdash T\ t : [t/x] K}
        { \Gamma \vdash T : \abs{\Pi}{x}{T'}{K}
        \quad \Gamma \decchk t : T' }
    \end{array}
  \]
  \caption{Sort checking \fbox{$\Gamma \vdash C : S$}}
  \label{fig:sort-checking}
\end{figure}

\begin{figure}[h!]
  \[ \small
    \begin{array}{lcr}
      \infer
      { \Gamma \decdir u : \Gamma(u)}{}
      & \infer
        { \Gamma \decdir \abs{\lambda}{x}{T}{t} : \abs{\Pi}{x}{T}{T'}}
        { \Gamma \vdash T : K
        \quad \Gamma, \ann{x}{T} \decdir t : T'}
      & \infer
        { \Gamma \decchk \absu{\lambda}{x}{t} : \abs{\Pi}{x}{T}{T'}}
        { \Gamma, \ann{x}{T} \decchk t : T'}
      \\
      \\ \infer
      { \Gamma \decdir \abs{\Lambda}{x}{C}{t} : \abs{\forall}{x}{C}{T}}
      { \Gamma \vdash C : S
      \quad x \notin FV(|t|)
      \quad \Gamma, \ann{x}{C} \decdir t : T
      }
      & \infer
        { \Gamma \decchk \absu{\Lambda}{x}{t} : \abs{\forall}{x}{C}{T}}
        { x \notin FV(|t|)
        \quad \Gamma, \ann{x}{C} \decdir t : T
        }
      & \infer
        { \Gamma \decdir t\ t' : [t'/x]T}
        { \Gamma \decsyn t : \abs{\Pi}{x}{T'}{T}
        \quad \Gamma \decchk t' : T'}
      \\
      \\ \infer
      { \Gamma \decdir t\ \cdot T : [T/X]T'}
      { \Gamma \decsyn t : \abs{\forall}{X}{K}{T'}
      \quad \Gamma \vdash T : K}
      & \infer
        { \Gamma \decdir t\ \mhyph t' : [t'/x]T}
        { \Gamma \decsyn t : \abs{\forall}{x}{T'}{T}
        \quad \Gamma \decchk t' : T'}
      & \infer % conversion... maybe needs to include phi and rho now?
        { \Gamma \decchk t : T }
        { \Gamma \decsyn t : T'
          & |T'| =_{\beta} |T| }
      \\ \\ \infer
      { \Gamma \decdir [ id : T = t ]\ \mhyph\ t' : T'}
      { \Gamma \vdash T : K
        & \Gamma \decchk t : T
        & \Gamma, \ann{id = t}{T} \decdir t' : T'}
      & \infer
        { \Gamma \decdir [ id = t]\ \mhyph\ t' : T' }
        { \Gamma \decsyn t : T
          & \Gamma, \ann{id = t}{T} \decdir t' : T'
        }
      & \infer[\footnotemark] % TODO
        { \Gamma \decdir \rho\ t\ \mhyph\ t' : [t_2/x]\ T}
        { \Gamma \decsyn t : \{ t_1 \simeq t_2 \}
          & \Gamma \decsyn t' : [t_1/x]\ T
        }
      \\ \\ \infer
      { \Gamma \decdir [ id : K = T ]\ \mhyph\ t' : T'}
      { \Gamma \vdash K : \square
        & \Gamma \vdash T : K
        & \Gamma, \ann{id = T}{K} \decdir t' : T'}
      & \infer
        { \Gamma \decchk \beta \{t\} : \{ t' \simeq t' \}}
        { \Gamma \vdash \{ t' \simeq t' \} : \star }
      & \infer
        { \Gamma \decdir \varsigma\ t : \{ t_2 \simeq t_1 \} }
        { \Gamma \decdir t : \{ t_1 \simeq t_2 \}}
      \\ \\ \infer
      { \Gamma \decdir \phi\ t\ \mhyph\ t_1\ \{t_2\} : T}
      { \Gamma \decchk t : \{ |t_1| \simeq |t_2| \}
        & \Gamma \decdir t_1 : T}
      & \infer
        { \Gamma \decsyn \chi\ T\ \mhyph\ t : T }
        { \Gamma \decchk t : T }
      & \infer[\footnotemark]
        { \Gamma \decchk \delta\ \mhyph\ t : T }
        { \Gamma \decchk t : \{ \texttt{tt}\ \simeq\ \texttt{ff} \}}
      \\ \\ \infer
      { \Gamma \decchk \theta\ t\ t'^* : T }
      { \Gamma \decsyn t : \texttt{??}
        & \Gamma \vdash_? t'^* : \texttt{??} }
    \end{array}
  \]
  \caption{Type checking \fbox{$\Gamma \decdir s : C$} (sans inductive datatypes)}
  \label{fig:type-checking}
\end{figure}
\footnotetext{Where we assume $t$ does not occur anywhere in $T$}
\footnotetext{Where $\texttt{tt} = \absu{\lambda}{x}{\absu{\lambda}{y}{x}}$ and
  $\texttt{ff} = \absu{\lambda}{x}{\absu{\lambda}{y}{y}}$}
% TODO kind-variables... two different rules or Var check?
% TODO equality, now that it can have rho, phi

The inference rules for classifying expressions in Cedilleum are stratified into
two judgments. Figure \ref{fig:sort-checking} gives the uni-directional rules
for ensuring types are well-kinded and kinds are well-formed. Future versions of
Cedilleum will allow for bidirectional checking for both typing \textit{and}
sorting, allowing for a unification of these two figures. Most of these rules
are similar to what one would expect from the Calculus of Implicit
Constructions, so we focus on the typing rules unique to Cedilleum.

The typing rule for ρ shows that ρ is a primitive for rewriting by an (untyped)
equality. If $t$ is an expression that synthesizes a proof that two terms $t_1$
and $t_2$ are equal, and $t'$ is an expression synthesizing type $[t_1/x]\ T$
(where, as per the footnote, $t_1$ does not occur in $T$), then we may
essentially rewrite its type to $[t_2/x]\ T$. The rule for β is reflexivity for
equality -- it witnesses that a term is equal to itself, provided that the type
of the equality is well-formed. The rule for ς is symmetry for equality.
Finally, φ acts as a ``casting'' primitive: the rule for its use says that if
some term $t$ witnesses that two terms $t_1$ and $t_2$ are equal, and $t_1$ has
been judged to have type $T$, then intuitively $t_2$ can also be judged to have
type $T$. (This intuition is justified by the erasure rule for φ -- the
expression erases to $|t_2|$). The last rule involving equality is for δ, which
witnesses the logical principle \textit{ex falso quodlibet} -- if a certain
impossible equation is proved (namely that the two Church-encoded booleans
\texttt{tt} and \texttt{ff} are equal), then \textit{any} type desired is
inhabited.

The two remaining primitives are not essential to the theory but are useful
additions for programmers. The rule for χ allows the user to provide an
explicit top-level annotation for a term, and θ embodies ``elimination with a
motive'', using the expected type of an application to infer some type
arguments. (TODO)

\section{Inductive Datatypes}

Before we can provide the typing rules for introduction and usage of inductive
datatypes, some auxiliary definitions must be given. The syntax for these, and
the structure of this entire section, borrows heavily from the conventions of the Coq
documentation\footnote{https://coq.inria.fr/refman/language/cic.html\#inductive-definitions}.
The author believes it is worthwhile to restate this development in terms of the
Cedilleum type system, rather than merely pointing readers to the Coq
documentation and asking them to infer the differences between the two systems.

To begin with, the production $defDataType$ gives the concrete syntax for datatype definitions,
but it is not a very useful notation for representing one in the abstract syntax
tree. In our typing rules we will instead use the notation
$\indast{M}{p}{\Gamma_I}{\Sigma}$, where

\begin{itemize}
\item $M$ is a meta-variable ranging over
  constant labels ``C'' and ``A'' (used to distinguish \textbf{c}oncrete and
  \textbf{a}bstracted inductive definitions -- more on this below)
\item $p$ is the number of \textbf{p}arameters of the inductive definition
\item $\Gamma_I$ is a typing context binding \textit{one} type variable $I$, the
  inductive type being defined
\item $\Sigma$ is a typing context containing the $n$ data constructors
  $c_1,...,c_n$ of $I$.
\end{itemize}

For example, consider the \texttt{List} and \texttt{Vec} definitions from
Section \ref{sec:syntax}. These will be represented in the AST as
\\ \\
\[\indast{\text{C}}{1}{List : ★ ➔ ★}
{\begin{array}{lcl}
   nil & : & ∀ A : ★ . List \cdot A
   \\ cons & : & ∀ A : ★ . A ➔ List \cdot A ➔ List \cdot A
 \end{array}
}\] and
\\
\[\indast{\text{C}}{1}{Vec : ★ ➔ Nat ➔ ★}
{\begin{array}{lcl}
   vnil & : & ∀ A : ★ . ∀ n : Nat . \{ n ≃ zero \} ➾ Vec \cdot A\ n
   \\ vcons & : & ∀ A : ★ . ∀ n : Nat . ∀ m : Nat . A ➔ Vec \cdot A\ n ➔ \{ m ≃ succ\ n \} ➾ Vec \cdot A\ m
 \end{array}
}\]

\noindent All inductive types the user will define will be concrete inductive
defintions, and have global scope. Abstracted definitions are automiatically
generated during fix-point pattern matching, and have local scope.

For an inductive datatype definition to be well-formed, it must satisfy the
following conditions (each of which is explained in more detail in Subsections
\ref{ssec:inductive-aux-defs} and \ref{ssec:inductive-wf-def}):

\begin{itemize}
\item The kind of $I$ must be (at least) a \textit{p-arity of kind ★}.
\item The types of each $id \in \Sigma$ must be \textit{types of constructors
    of $I$}
\item The definition must satisfy the \textit{non-strict} positivity condition.
\end{itemize}

Similarly, the notation in the grammar of Cedilleum $\mu'$ and $\mu$ for pattern
matching is inconvenient, and we will represent them in the AST as resp.
$\mu'(t,P,t_{i=1..n})$ and
$\mu(x_{\text{rec}},I',x_{\text{to}},t,P,t_{i=1..n})$. Translation from the form
given in the grammar to this form is discussed in detail below, but is as
expected. In particular, we enforce that patterns are exhaustive and
non-overlapping, and that $I'$ and $x_{\text{to}}$ (which are not present in the grammar
but an automatically generated identifier) are fresh w.r.t the global and local
context. For example, consider the pattern-matches given in the code listings
for \texttt{isvnil} and \texttt{vlength} above. These would be translated into
the AST as
\\ \\
\[
  \mu'(xs,\absu{\Lambda}{n}{\absu{\lambda}{x}{Bool}},
  \begin{array}{l}
    \absu{\Lambda}{n}{\absu{\Lambda}{eq}{tt}}
    \\ \absu{\Lambda}{n}{\absu{\Lambda}{m}{\absu{\lambda}{x}{\absu{\lambda}{xs}{\absu{\Lambda}{eq}{ff}}}}}
  \end{array}
  )
\] and
\[ \mu(len, Vec/len, toVec/len,xs,\absu{\Lambda}{n}{\absu{\lambda}{x}{Nat}},
  \begin{array}{l}
    \absu{\Lambda}{n}{\absu{\Lambda}{eq}{zero}}
    \\ \absu{\Lambda}{n}{\absu{\Lambda}{m}{\absu{\lambda}{x}{\absu{\lambda}{xs}{\absu{\Lambda}{eq}{suc\
    (len\ \mhyph n\ xs)}}}}}
  \end{array})
\]

\noindent In general, the generated name for $I'$ and $x_{\text{to}}$ that users
will write in Cedilleum programs will be of the form ``$I\text{/}x_{\text{rec}}$''
and ``$\text{to}I\text{/}x_{\text{rec}}$''.

For a pattern construct ($\mu$ or $\mu'$) in the AST to be well-formed, it must satisfy the
following conditions (each of which is, again, explained in more detail in
Subsections \ref{ssec:pattern-valid-elim}, \ref{ssec:pattern-wf-pat}, and
\ref{ssec:patern-abstracted-gen}):

\begin{itemize}
\item The motive $P$ must be well-kinded
\item $P$ must be a legal motive to be used in eliminating the inductive type
  $I$ of the scrutinee $t$
\item Each branch $t_i$ must have the type expected given the constructor $c_i
  \in \Sigma$ and the motive $P$.
\end{itemize}

\subsection{Auxiliary Definitions}
\label{ssec:inductive-aux-defs}

\paragraph{Contexts}
To ease the notational burden, we will introduce some conventions for writing
contexts within terms and types.

\begin{itemize}
\item We write $\lambda\,\Gamma$, $\Lambda\,\Gamma$, $\forall\,\Gamma$, and
  $\Pi\,\Gamma$ to indicate some form of abstraction over each variable in
  $\Gamma$. For example, if $\Gamma = \ann{x_1}{T_1},\ann{x_2}{T_2}$ then
  $\absu{\lambda}{\Gamma}{t} =
  \abs{\lambda}{x_1}{T_1}{\abs{\lambda}{x_2}{T_2}{t}}$. Additionally, we
  will also write $\piforall\,\Gamma$ to indicate an arbitrary mixture of $\Pi$
  and $\forall$ quantified variables. Note that \textit{if $\piforall\,\Gamma$
  occurs multiple times within a definition or inference rule}, the intended
  interpretation is that \textit{all occurrences have the same mixture of $\Pi$
    and $\forall$ quantifiers}.
\item $\lenc{\Gamma}$ denotes the length of $\Gamma$ (the number of variables it
  binds)
\item We write $s\ \Gamma$ to indicate the sequence of variable arguments in
  $\Gamma$ given as arguments to $s$. Implicit in this notation is the removal
  of typing annotations from the variables $\Gamma$ when these variables are
  given as arguments to $s$.

  Since in Cedilleum there are three flavors of applications (to a type, to an
  erased term, and to an unerased term), we will only us this notion when the type
  or kind of $s$ is known, which is sufficient to disambiguate the flavor of
  application intended for each particular binder in $\Gamma$. For example,
  if $s$ has type
  $\abs{\forall}{X}{★}{\abs{\forall}{x}{X}{\abs{\Pi}{x'}{X}{X}}}$ and $\Gamma =
  \ann{X}{★},\ann{x}{X},\ann{x'}{X}$ then $s\ \Gamma = s\ \cdot X\ \mhyph x\ x'$
\item $\Delta$ and $\Delta'$ are notations we will use
  for a specially designated contexts associating type variables with both global
  ``concrete'' and local ``abstracted'' inductive data-type declarations.
  The purpose of this latter sort of declaration is to enable type-guided
  termination of definitions using fixpoints (see Section \ref{ssec:typing-rules}) For example, given
  just the (global) data type declaration of $Vec$, we would have $\Delta(Vec) =
  \indast{\text{C}}{1}{\Gamma_{Vec}}{\Sigma}$, where $\Gamma_{Vec} = \ann{Vec}{★ ➔ Nat ➔
    ★}$ and  $\Sigma$ binds data constructors $vnil$ and $vcons$ to the
  appropriate types.
\end{itemize}

\paragraph{$p$-arity}

A kind $K$ is a $p$-arity if it can be written as $\absu{\Pi}{\Gamma}{K'}$ for
some $\Gamma$ and $K'$, where $\lenc{\Gamma} = p$. For an inductive definition
$\indast{M}{p}{\Gamma_I}{\Sigma}$, requiring that the kind $\Gamma_{I}(I)$ is a $p$-arity
of ★ ensures that $I$ \textit{really does have} $p$ parameters.

\paragraph{Types of Constructors}
% TODO: if you look at the `generation of abstracted inductive definitions', it
% uses a different format for the types associated with the constructors in
% \Sigma -- that is the \piforall notation. This section probably should be
% reworked to that end.
$T$ is a \textit{type of a constructor of $I$} iff
\begin{itemize}
\item it is $I\ s_1 ... s_n$
\item it can be written as $\abs{\forall}{s}{C}{T}$ or $\abs{\Pi}{s}{C}{T}$,
  where (in either case) $T$ is a type of a constructor of $I$
\end{itemize}

\paragraph{Positivity condition}
The positivity condition is defined in two parts: the positivity condition of
a type $T$ of a constructor of $I$, and the positive occurence of $I$ in $T$.
We say that a type $T$ of a constructor of $I$ satisfies the positivity condition
when

\begin{itemize}
\item $T$ is $I\ s_1... s_n$ and $I$ does not occur anywhere in $s_1...s_n$
\item $T$ is $\abs{\forall}{s}{C}{T'}$ or $\abs{\Pi}{s}{C}{T'}$, $T'$ satisfies
  the positivity condition for $I$, and $I$ occurs \textit{only} positively in $C$ 
\end{itemize}

\noindent We say that $I$ occurs only positively in $T$ when
\begin{itemize}
\item $I$ does not occur in $T$
\item $T$ is of the form $I\ s_1 ... s_n$ and $I$ does not occur in $s_1 ...
  s_n$
\item $T$ is of the form $\abs{\forall}{s}{C}{T'}$ or $\abs{\Pi}{s}{C}{T'}$, $I$
  occurs only positively in $T'$, and $I$ \textit{does not} occur positively in $C$
\end{itemize}

\subsection{Well-formed inductive definitions}
\label{ssec:inductive-wf-def}

Let $\Gamma_{\text{P}},\Gamma_I,$ and $\Sigma$ be contexts such that $\Gamma_I$
associates a single type-variable $I$ to kind $\absu{\Pi}{\Gamma_{\text{p}}}{K}$ and
$\Sigma$ associates term variables $c_1 ... c_n$ with corresponding types
$\absu{\forall}{\Gamma_{\text{P}}}{T_{1}},...\absu{\forall}{\Gamma_{\text{P}}}{T_{n}}$.
Then the rule given in Figure \ref{fig:inductive-intro} states when an inductive
datatype definition may be introduced, provided that the following side
conditions hold:

\begin{figure}[h]
  \caption{Introduction of inductive datatype}
  \label{fig:inductive-intro}
  \[
    \infer
    { \indast{M}{p}{\Gamma_I}{\Sigma}\ wf}
    { \emptyset \vdash \Gamma_I(I) : \square
      \quad \lenc{\Gamma_P} = p
      \quad (\Gamma_I,\Gamma_P \vdash T_i : ★)_{i=1..n}
    }
  \]
\end{figure}

\begin{itemize}
  \item Names $I$ and $c_1..c_n$ are distinct from any other inductive datatype
    type or constructor names, and distinct amongst themselves
  \item Each of $T_1..T_n$ is a type of constructor of $I$ which satisfies the
    positivity condition for $I$. Furthmore, each occurence of $I$ in $T_i$ is
    one which is applied to the parameters $\Gamma_P$.
  \item Identifiers $I$, $c_1,...,c_n$ are fresh w.r.t the global context, and
    do not overlap with each other nor any identifiers in $\Gamma_P$.
\end{itemize}

When an inductive data-type has been defined using the $defDataType$ production,
it is understood that this always a concrete inductive type, and it (implicitly)
adds to a global typing context the variable bindings in $\Gamma_I$ and
$\Sigma$. Similarly, when checking that the kind $\Gamma_I(I)$ and type $T_i$
are well-sorted and well-kinded, we assume an (implicit) global context of
previous definitions.

\subsection{Valid Elimination Kind}
\label{ssec:pattern-valid-elim}

\begin{figure}[h]
  \caption{Valid elimination kinds}
  \label{fig:valid-elim-kind}
  \[
    \begin{array}{ccc}
      \infer
      { \llbracket T : ★\ |\ T \to ★ \rrbracket }
      { }
      & \infer
        { \llbracket T : \abs{\Pi}{s}{C}{K}\ |\ \abs{\Pi}{s}{C}{K'} \rrbracket}
        { \llbracket T\ s : K\ |\ K' \rrbracket }
    \end{array}
  \]
\end{figure}

When type-checking a pattern match (either $\mu$ or $\mu'$), we need to know
that the given motive $P$ has a kind $K$ for which elimination of a term with
some inductive data-type $I$ is permissible. We write this judgment as
$\llbracket \ann{T}{K'} | K \rrbracket$, which should be read ``the type $T$ of kind $K'$ can
be eliminated through pattern-matching with a motive of kind $K$''. This
judgment is defined by the simple rules in Figure \ref{fig:valid-elim-kind}. For
example, a valid elimination kind for the indexed type family $Vec\ \cdot X$
(which has kind $\abs{\Pi}{n}{Nat}{★}$) is $\abs{\Pi}{n}{Nat}{\abs{\Pi}{x}{Vec\
    \cdot X\ n}{★}}$

\subsection{Valid Branch Type}

Another piece of kit we need is a way to ensure that, in a pattern-matching
expression, a particular branch has the correct type given a particular
constructor of an inductive data-type and a motive. We write $\llbrace c : T
\rrbrace^P_I$ to indicate the type corresponding to the (possibly partially
applied) constructor $c$ of $I$ and its type $T$. We
abbreviate this notation to $\llbrace c \rrbrace^P$ when the inductive type
variable $I$, and the type $T$ of $c$, is known from the (meta-language) context.

\[
  \begin{array}{rcl}
    \llbrace c : I\ \vars{T}\ \vars{s} \rrbrace^P_I
    & = & P\ \vars{s}\ c
    \\ \llbrace c : \abs{\forall}{x}{T'}{T} \rrbrace^P_I
    & = & \abs{\forall}{x}{T'}{\llbrace c\ \mhyph x : T \rrbrace^P_I }
    \\ \llbrace c : \abs{\forall}{x}{K}{T} \rrbrace^P_I
    & = & \abs{\forall}{x}{K}{\llbrace c\ \cdot x : T \rrbrace^P_I }
    \\ \llbrace c : \abs{\Pi}{x}{T'}{T} \rrbrace^P_I
    & = & \abs{\Pi}{x}{T'}{\llbrace c\ x : T \rrbrace^P_I }
  \end{array}
\]

\noindent where we leave implicit the book-keeping required to separate the
parameters $\vars{T}$ from the indicies $\vars{s}$.

The biggest difference bewteen this definition and the similar one found in the
Coq documentation is that types can have implicit and explicit quantifiers, so
we must make sure that the types of branches have implicit / explicit
quantifiers (and the subjects $c$ have applications for types, implicit terms, and
explicit terms), corresponding to those of the arguments to the data constructor
for the pattern for the branch.

\subsection{Well-formed Patterns}
\label{ssec:pattern-wf-pat}

\begin{figure}[h]
  \caption{Well-formedness of a pattern}
  \label{fig:wf-pattern}
  \[
    \infer
    { \wfpat{\Gamma,\Delta}{\indast{M}{p}{\Gamma_I}{\Sigma}}{\vars{T}}{\mu'(t,P,t_{i=1..n})}
    }
    { \Gamma \vdash P : K
      \quad \Sigma = \ann{c_1}{\absu{\forall}{\Gamma_P}{T_1}}, ..., \ann{c_n}{\absu{\forall}{\Gamma_P}{T_n}}
      \quad \lenc{\vars{T}} = \lenc{\Gamma_p} = p
      \quad \llbracket I\ \vars{T}\, : \Gamma(I)\, |\, K \rrbracket
      \quad (\Gamma,\Delta \decchk t_i : \llbrace c_i\ \vars{T} \rrbrace^P)_{i=1..n}
    }
  \]
\end{figure}

% TODO 
Figure \ref{fig:wf-pattern} gives the rule for checking that a pattern
$\mu'(t,P,t_{i=1..n})$ is well-formed. We check that the motive $P$ is
well-kinded at kind $K$, that the given parameters $\vars{T}$ match the expected
number $p$ from the inductive data-type declaration, that an inductive data-type
$I$ instantiated with the given parameters $\vars{T}$ can be eliminated to a
type of kind $K$, and that the given branches $t_i$ account for each of the
constructors $c_i$ of $\Sigma$ and have the required branch type $\llbrace c_i\
\vars{T} \rrbrace^P$ under the given local context $\Gamma$ and context of
inductive data-type declarations $\Delta$.

\subsection{Generation of Abstracted Inductive Definitions}
\label{ssec:patern-abstracted-gen}

Cedilleum supports \textit{histomorphic} recursion (that is, having access to
all previous recursive values) where termination is ensured
through typing. In order to make this possible, we need a mechanism for tracking
the global definitions of \textit{concrete} inductive data types as well the
locally-introduced \textit{abstract} inductive data type representing the
recursive occurences suitable for a fixpoint function to be called on.

If $I$ is an inductive type such that $\Delta(I) =
\indast{\text{C}}{p}{\Gamma_I}{\Sigma}$ and $I'$ is a fresh type variable, then we
define function $Hist(\Delta,I,\vars{T},I')$ producing an abstracted (well-formed)
inductive definition $\indast{\text{A}}{0}{\Gamma_{I'}}{\Sigma'}$, where

\begin{itemize}
\item $\Gamma_{I'}(I') = \absu{\forall}{\Gamma_D}{★}$ if $\Gamma_{I}(I) =
  \absu{\forall}{\Gamma_{P}}{\absu{\forall}{\Gamma_D}{★}}$ (and $\lenc{\Gamma_P}
  = \lenc{\vars{T}} = p$)

  That is, the kind of $I'$ is the same as the kind of $I\ \vars{T}$
\item $\Sigma' = \ann{c'_1}{\absu{\forall}{\Gamma_D}
    { \absu{\piforall}{\Gamma_{A'_1}}{I'\ \Gamma_D} }},...,
  \ann{c'_n}{\absu{\forall}{\Gamma_D}
    { \absu{\piforall}{\Gamma_{A'_n}}{I\ \vars{T}\ \Gamma_D} }}$,

  when each of the concrete constructors $c_i$ in $\Sigma$ are associated with
  type $\absu{\forall}{\Gamma_P}{
    \absu{\forall}{\Gamma_D}{ \absu{\piforall}{\Gamma_{A_i}}{I\ \Gamma_P\
        \Gamma_D} } }$ and each $\Gamma_{A'_i} =
  [\absu{\lambda}{\Gamma_P}{I'}/I,\vars{T}/\Gamma_P]\Gamma_{A_i}$.

  That is, trasforming the concrete constructors of the inductive datatype $I$
  to ``abstracted'' constructors involves replacing each recursive occurrence of
  $I\ \Gamma_P$ with the fresh type variable $I$, and instantiating each of the
  parameters $\Gamma_P$ with $\vars{T}$.
\end{itemize}

Users of Cedilleum will see ``punning'' of the concrete constructors $c_i$ and
abstracted constructors $c'_i$. In particular, when using fix-point pattern
matching branch labels will be written with the constructors for the concrete
inductive data-type, and the expected type of a branch given by the motive will
pretty-print using the concrete constructors. In the inference rules, however,
we will take more care to distinguish the abstract constructors (see Subsection
\ref{ssec:typing-rules}).

\subsection{Typing Rules}
\label{ssec:typing-rules}

\begin{figure}[h]
  \caption{Use of an inductive datatype $\indast{M}{p}{\Gamma_I}{\Sigma}$}
  \label{fig:inductive-use}
  \[ \footnotesize
    \begin{array}{c}
      \infer
      { \Gamma,\Delta \decdir \mu'(t,P,t_{i=1..n}) : P\ \vars{s}\ t}
      { \Gamma \decsyn t : I\ \vars{T}\ \vars{s}
      \quad \wfpat{\Gamma,\Delta}{\Delta(I)}{\vars{T}}{\mu'(t,P,t_{i=1..n})}
      }
      \\ \\
      \\ \infer
      { \Gamma,\Delta \decdir \mu(x_{\text{rec}}, I',
      x_{\text{to}},t,P,t_{i=1..n}) : P\ \vars{s}\ t
      }
      {
      \begin{array}{c}
        \begin{array}{cccc}
          \Gamma \decsyn t : I\ \vars{T}\ \vars{s}
          & \Delta(I) = \indast{\text{C}}{p}{\Gamma_I}{\Sigma}
          & \Gamma_I(I) =
            \absu{\Pi}{\Gamma_P}{\absu{\Pi}{\Gamma_{\text{D}}}{★}},\lenc{\Gamma_P}
            = p
          & Hist(\Delta,I,\vars{T},I') = \indast{\text{A}}{0}{\Gamma_{I'}}{\Sigma'}
        \end{array}
        \\ \\
        \begin{array}{cc}
          \Gamma' = \Gamma,\Gamma_{I'},
          \ann
           {x_{\text{to}}=\absu{\Lambda}{\Gamma_D}{\absu{\lambda}{x}{x}}}
           { \absu{\forall}{\Gamma_{\text{D}}}{I'\
            \Gamma_{\text{D}} \to I\ \vars{T}\
            \Gamma_{\text{D}}}},
          \ann{x_{\text{rec}}}{\absu{\forall}{\Gamma_{\text{D}}}{\abs{\Pi}{x}{I'\
          \Gamma_{\text{D}}}{P\ \Gamma_{\text{D}}\ (x_{\text{to}}\ \Gamma_D\ x)}
          }}
          & \Delta' = \Delta,Hist(\Delta,I,\vars{T},I')
        \end{array}
        \\ \\
        \begin{array}{cc}
          % P' = \absu{\lambda}{\Gamma_D}{\abs{\lambda}{x}{I\ \vars{T}\ \Gamma_D}{P\ 
          % \Gamma_D\ x} }
          \wfpat{\Gamma',\Delta'}{\Delta'(I')}{\varnothing}{\mu'(t,P,t_{i=1..n})}
        \end{array}
      \end{array}
      }
    \end{array}
  \]
\end{figure}

The first rule of Figure \ref{fig:inductive-use} is for typing simple pattern
matching with $\mu'$. We need to know that the scrutinee $t$ is well-typed at
some inductive type $I\ \vars{T}\ \vars{s}$, where $\vars{T}$ represents the
parameters and $\vars{s}$ the indicies. Then we defer to the judgment
$WF\mhyph\!Pat$ to ensure that this pattern-matching expression is a valid
elimination of $t$ to type $P$.

The second rule is for typing pattern-matching with fix-points, and is
significantly more involved. As above we check the scrutinee $t$ has some
inductive type $I\ \vars{T}\ \vars{s}$. We confirm that $I$ is a
\textit{concrete} inductive data-type by looking up its definition in $\Delta$,
and then generate the abstracted definition $Hist(\Delta,I,\vars{T},I')$ for some fresh
$I'$. We then add to the local typing context $\Gamma_{I'}$ (the new inductive
type $I'$ with its associated kind) and two new variables $x_{\text{to}}$ and
$x_{\text{rec}}$.

\begin{itemize}
\item $x_{\text{to}}$ is the \textit{revealer}. It casts a term of an abstracted inductive
  data-type $I'\ \Gamma_D$ to the concrete type $I\ \vars{T}\ \Gamma_D$.
  Crucially, it is an \textit{identity} cast (the implicit quantification
  $\Lambda \Gamma_D$ disappears after erasure). The intuition why this should be
  the case is that the abstracted type $I'$ only serves to mark the recursive
  occurrences of $I$ during pattern-matching to guarantee termination.
\item $x_{\text{rec}}$ is the \textit{recursor} (or the inductive hypothesis).
  Its result type $P'\ \Gamma_D\ x$ utilizes $x_{\text{to}}$ in $P'$ to be
  well-typed, as the $x$ in this expression has type $I'\ \Gamma_D$, but $P$
  expects an $I\ \vars{T}\ \Gamma_D$. Because $x_{\text{to}}$ erases to the identity, uses of the
  $x_{\text{rec}}$ will produce expressions whose types will not interfere with
  producing the needed result for a given branch (see the extended example --
  TODO).
\end{itemize}

\noindent With these definitions, we finish the rule by checking that the
pattern is well-formed using the augmented local context $\Gamma'$ and context
of inductive data-type definitions $\Delta'$.

\end{document}
